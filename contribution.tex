\documentclass{scdpg}
\begin{document}
\scBookLanguage{en}
\begin{scAbstract}
%\scNoUseTeX
\scLanguage{en}
\scTitle{Proton range calibration for the R\textsuperscript{3}B-CALIFA calorimeter}
\scAuthor{*}{Mrunmoy}{Jena}{1}
\scAuthor{}{Roman}{Gernh\"auser}{1}
\scAuthor{}{Tobias}{Jenegger}{1}
\scAffiliation{1}{Technische Universit\"at M\"unchen}
\scBeginText
The CALIFA (CALorimeter for In-Flight detection of gamma rays and high energy charged pArticles) is one of the most important detectors in the R\textsuperscript{3}B (Reactions with Relativistic Radioactive ion Beams) experiment. Being highly segmented and having almost full polar angle coverage ($7^{o} < \theta < 143^{o}$), this detector provides spectroscopic information for gamma rays and charged particle energies varying from 100 keV to about 300 MeV. The MPRB-32 charge sensitive preamplifiers coupled to the detection units can be operated in a low gain (gamma range) or a high gain mode (proton range), enabling a high dynamic range for the energy determination.

This presentation introduces a user-friendly, computer-controlled procedure that carries out an automatic calibration of the entire CALIFA calorimeter over the full dynamic range. The calibration is carried out using a combination of a $^{22}\mathrm{Na}$ radioactive source and electronic pulser signals of known amplitudes.
\newline
Supported by BMBF 05P24WO2.
\scEndText
\scConference{K\"oln 2025}
\scPart{HK}
\scContributionType{Vortrag;Talk}
\scTopic{Struktur und Dynamik von Kernen}
\scKeywords{CALIFA, R3B, calibration}
\scEmail{mrunmoy.jena@tum.de}
\scCountry{Germany}
\end{scAbstract}
\end{document}
